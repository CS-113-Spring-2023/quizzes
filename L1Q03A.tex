\documentclass[addpoints]{exam}

\usepackage{amsmath,amssymb,amsthm}
\usepackage{tabularx}

\theoremstyle{definition}
\newtheorem{definition}{Definition}[section]

\theoremstyle{claim}
\newtheorem{claim}{Claim}

\title{Quiz 3A: Rules of Inference}
\author{CS/MATH 113 Discrete Mathematics L1}
\date{Habib University | Spring 2023}

% solution
\usepackage{draftwatermark}
\SetWatermarkText{Sample Solution}
\SetWatermarkScale{3}
\printanswers

\newcolumntype{L}{>$l<$}

\begin{document}
\maketitle
\thispagestyle{empty}

\noindent
\begin{tabularx}{\linewidth}{Xr}
  Total Marks: \numpoints & Date: \today\\
  Duration: 10 minutes & Time: 1715--1725h
\end{tabularx}
\hrule
\bigskip

\noindent \textbf{Student ID}: \hrulefill \\[5pt]
\noindent \textbf{Student Name}: \hrulefill \\[5pt]

% \section{Problems}

\begin{questions}
  \question  [10] Use the rules of inference given below to determine whether the following argument is valid.
  \begin{quotation}
If Superman were able and willing to prevent evil, he would do so. If Superman were unable to prevent evil, he would be impotent; if he were unwilling to prevent evil, he would be malevolent. Superman does not prevent evil. If Superman exists, he is neither impotent nor malevolent. Therefore, Superman does not exist.    
  \end{quotation}
  \[
    \begin{array}{cc||cc||cc}
      \begin{array}{l}
        p\\p\implies q\\\hline\therefore q
      \end{array}
      &
        \text{Modus ponens}& 
      \begin{array}{l}
        \neg q\\p\implies q\\\hline\therefore \neg p
      \end{array}
      &
        \text{Modus tollens}& 
      \begin{array}{l}
        p\implies q\\q\implies r\\\hline\therefore q\implies r
      \end{array}
      &
        \text{Hypothetical syllogism}\\
      \hline\hline
      \begin{array}{l}
        p\lor q\\\neg p\\\hline\therefore q
      \end{array}
      &
        \text{Disjunctive syllogism}&
      \begin{array}{l}
        p\\\hline\therefore p\lor q
      \end{array}
      &
        \text{Addition}& 
      \begin{array}{l}
        p\land q\\\hline\therefore p
      \end{array}
      &
        \text{Simplification}\\
      \hline\hline
      \begin{array}{l}
        p\\ q\\\hline\therefore p\land q
      \end{array}
      &
        \text{Conjunction}& 
      \begin{array}{l}
        p\lor q\\ \neg p\lor r\\\hline\therefore q\lor r
      \end{array}
      &
        \text{Resolution} 
    \end{array}
  \]
  
  \begin{solution}
    We argue that the given argument is valid by deriving the conclusion from the premises.
    \begin{proof}
    Let us define the following atomic propositions:    
    \begin{tabular}{Ll}
      a & Superman is able to prevent evil\\
      w & Superman is willing to prevent evil\\
      p & Superman prevents evil\\
      i & Superman is impotent\\
      m & Superman is malevolent\\
      e & Superman exists\\
    \end{tabular}

    The given argument can then be denoted as:
    \begin{tabular}{L@{\qquad}L}
        (a\land w) \implies p & P1\\
        \neg a \implies i & P2\\
        \neg w \implies m & P3\\
        \neg p & P4\\
        e \implies (\neg i \land \neg m) & P5\\
        \hline
        \therefore \neg e & C
    \end{tabular}
    
    And we can argue as follows:
    \begin{tabular}{L@{\quad}lL@{\qquad}L}
      \neg(a\land w) & Modus tollens & P1, P4 & P6\\
      \neg a \lor \neg w & DeMorgan's law & P6 & P7\\
      a \lor i & Implication & P2 & P8 \\
      w \lor m & Implication & P3 & P9 \\
      \neg w \lor i & Resolution & P7, P8 & P10 \\
      m \lor i & Resolution & P9, P10 & P11 \\
      \neg(\neg i \land \neg m) \implies \neg e & Contrapositive & P5 & P12\\
      (i \lor m) \implies \neg e & DeMorgan's law & P5 & P13\\
      \neg e & Modus ponens & P12, P13 & P14\\
    \end{tabular}

    $P14$ is the same as $C$. Therefore the argument is valid. 
  \end{proof}
\end{solution}
\end{questions}
\end{document}
