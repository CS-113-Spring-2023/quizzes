\documentclass[addpoints]{exam}

\usepackage{amsmath,amssymb,amsthm}
\usepackage{tabularx}
\usepackage[table]{xcolor}

\theoremstyle{definition}
\newtheorem{definition}{Definition}[section]

\theoremstyle{claim}
\newtheorem{claim}{Claim}

\title{Quiz 2A: Logical Equivalence}
\author{CS/MATH 113 Discrete Mathematics L1}
\date{Habib University | Spring 2023}

% solution
\usepackage{draftwatermark}
\SetWatermarkText{Sample Solution}
\SetWatermarkScale{3}
\printanswers

\begin{document}
\maketitle
\thispagestyle{empty}

\noindent
\begin{tabularx}{\linewidth}{Xr}
  Total Marks: \numpoints & Date: \today\\
  Duration: 10 minutes & Time: 1715--1725h
\end{tabularx}
\hrule
\bigskip

\noindent \textbf{Student ID}: \hrulefill \\[5pt]
\noindent \textbf{Student Name}: \hrulefill \\[5pt]

\section{Problems}

\begin{questions}
  \question[5] We are given the following definitions.

\begin{definition}[Negation, $\neg$]
  \[
  \begin{array}{c||c}
    p & \neg p\\
    \hline
    T & F \\
    F & T \\
  \end{array}
  \]
\end{definition}

\begin{definition}[Conjunction, $\land$]
  \[
  \begin{array}{c|c||c}
    p & q & p \land q\\
    \hline
    T & T & T \\
    T & F & F \\
    F & T & F \\
    F & F & F \\
  \end{array}
  \]
\end{definition}

\begin{definition}[Disjunction, $\lor$]
  \[
  \begin{array}{c|c||c}
    p & q & p \lor q\\
    \hline
    T & T & T \\
    T & F & T \\
    F & T & T \\
    F & F & F \\
  \end{array}
  \]
\end{definition}

Use these to prove \textit{De Morgan's laws}, i.e.
\begin{parts}
  \part $\neg(p \land q) \equiv \neg p \lor \neg q $
  \part $\neg(p \lor q) \equiv \neg p \land \neg q $
\end{parts}
  
\begin{solution}
  We show that the columns for the LHS and RHS of {\color{olive}part (a)} are identical, and also of {\color{blue}part (b)}.
  \begin{proof}
    \small
  \[
  \begin{array}{c|c||c|c|c|c|>{\columncolor{olive!20}}c|>{\columncolor{blue!20}}c|>{\columncolor{olive!20}}c|>{\columncolor{blue!20}}c}
    p & q & \neg p & \neg q & p \land q & p \lor q & \neg(p \land q) & \neg(p \lor q) & \neg p \lor \neg q & \neg p \land \neg q\\
    \hline
    T & T & F & F & T & T & F & F & F & F \\
    T & F & F & T & F & T & T & F & T & F \\
    F & T & T & F & F & T & T & F & T & F \\
    F & F & T & T & F & F & T & T & T & T \\
  \end{array}
\]
    \end{proof}
  \end{solution}
\end{questions}
\end{document}
