\documentclass[addpoints]{exam}

\usepackage{amsmath,amssymb,amsthm}
\usepackage{tabularx}

\theoremstyle{definition}
\newtheorem{definition}{Definition}[section]

\theoremstyle{claim}
\newtheorem{claim}{Claim}

\title{Quiz 4B: Predicate logic}
\author{CS/MATH 113 Discrete Mathematics L1}
\date{Habib University | Spring 2023}

% solution
\usepackage{draftwatermark}
\SetWatermarkText{Sample Solution}
\SetWatermarkScale{3}
\printanswers

\newcolumntype{L}{>$l<$}

\begin{document}
\maketitle
\thispagestyle{empty}

\noindent
\begin{tabularx}{\linewidth}{Xr}
  Total Marks: \numpoints & Date: \today\\
  Duration: 10 minutes & Time: 1715--1725h
\end{tabularx}
\hrule
\bigskip

\noindent \textbf{Student ID}: \hrulefill \\[5pt]
\noindent \textbf{Student Name}: \hrulefill \\[5pt]

% \section{Problems}

\begin{questions}
  \question [10] The predicates $A(x), B(x)$, and $C(x)$ are defined on the domain $P=\{p_1,p_2,p_3,p_4\}$. The predicates $D(x), E(x)$, and $F(x)$ are defined on the domain $Q=\{q_1,q_2,q_3\}$. And the predicates $G(x,y), H(x,y)$ and  $I(x,y)$ are defined such that $x$ belongs to $P$ and $y$ to $Q$. Their truth tables are given below.
  \[
    \begin{array}{{c|ccc}}
      x & A(x) & B(x) & C(x) \\
      \hline
      p_1 & T & F & T \\
      p_2 & T & F & F \\
      p_3 & T & F & T \\
      p_4 & T & F & T \\
    \end{array}\
    \qquad
    \begin{array}{{c|ccc}}
      x & D(x) & E(x) & F(x) \\
      \hline
      q_1 & T & F & F \\
      q_2 & T & F & T \\
      q_3 & T & F & T \\
    \end{array}
    \qquad
    \begin{array}{{cc|ccc}}
      x & y & G(x,y) & H(x,y) & I(x,y)\\
      \hline
      p_1 & q_1 & T & F & F \\
      p_1 & q_2 & T & F & F \\
      p_1 & q_3 & T & F & F \\
      p_2 & q_1 & T & F & T \\
      p_2 & q_2 & T & F & T \\
      p_2 & q_3 & T & F & T \\
      p_3 & q_1 & T & F & F \\
      p_3 & q_2 & T & F & T \\
      p_3 & q_3 & T & F & F \\
      p_4 & q_1 & T & F & T \\
      p_4 & q_2 & T & F & F \\
      p_4 & q_3 & T & F & T \\
    \end{array}
  \]  
  Provide below some true quantified statements involving each of the predicates. For each statement, argue very briefly why it is true.
  
  \begin{solution}
    Many statements can be made. Below are a few.
    
    \underline{Involving $A(x), B(x)$, and $C(x)$}
    
    \begin{tabular}{L@{ : }l}
      \forall x\; A(x) & $A(x)$ is always True\\
      \forall x\; \neg B(x) & $B(x)$ is always False\\
      \forall x\; (A(x) \lor B(x)) & $A(x) \lor B(x)$ is always True\\
      \forall x\; \neg(A(x) \land B(x)) & $A(x) \land B(x)$ is always False\\
      \forall x\; \neg(A(x) \implies B(x)) & $A(x) \implies B(x)$ is always False\\
      \forall x\; (B(x) \implies A(x)) & $B(x) \implies A(x)$ is always True\\
      \forall x\; (A(x) \iff \neg B(x)) & $A(x)$ and $B(x)$ always contradict\\
      \forall x\; (B(x) \implies C(x)) & $B(x) \implies C(x)$ is always True\\
      \forall x\; (C(x) \implies A(x)) & $C(x) \implies A(x)$ is always True\\
      \forall x\; (A(x) \lor C(x)) & $A(x) \lor C(x)$ is always True\\
      \forall x\; \neg(B(x) \land C(x)) & $B(x) \land C(x)$ is always False\\
    \end{tabular}

    The existential counterparts of all the above statements also hold. In addition, the following existential statements also hold.

    \begin{tabular}{L@{ : }l}
      \exists x\; C(x)) & $C(x)$ is sometimes True\\
      \exists x\; \neg C(x)) & $C(x)$ is sometimes False\\
      \exists x\; (A(x) \iff C(x)) & $A(x)$ and $C(x)$ sometimes agree\\
      \exists x\; \neg (A(x) \iff C(x)) & $A(x)$ and $C(x)$ sometimes contradict\\
      \exists x\; (A(x) \iff B(x)) & $A(x)$ and $B(x)$ sometimes agree\\
      \exists x\; \neg (A(x) \iff B(x)) & $A(x)$ and $B(x)$ sometimes contradict\\
    \end{tabular}

    $D(x), E(x)$, and $F(x)$ can be reasoned about similarly.

    Following are some statements about $G(x,y), H(x,y)$, and $I(x,y)$.

    \begin{tabular}{L@{ : }l}
      \forall x\forall y\; G(x,y) & $G(x,y)$ is always True\\
      \forall x\forall y\; \neg H(x,y) & $H(x,y)$ is always False\\
      \forall x\forall y\; \neg( G(x,y)\iff  H(x,y)) & $G(x,y)$ and $H(x,y)$ always contradict\\
      \forall x\forall y\; ( G(x,y)\lor  H(x,y)) & $G(x,y) \lor H(x,y)$ is always True\\
      \forall x\forall y\; \neg ( G(x,y)\implies  H(x,y)) & $G(x,y) \implies H(x,y)$ is always False\\
      \forall x\forall y\; ( H(x,y)\implies  G(x,y)) & $H(x,y) \implies G(x,y)$ is always True\\
      \forall x\forall y\; ( H(x,y)\implies  I(x,y)) & $H(x,y) \implies I(x,y)$ is always True\\
      \forall x\forall y\; ( I(x,y)\implies  G(x,y)) & $I(x,y) \implies G(x,y)$ is always True\\
      \forall x\forall y\; \neg (H(x,y)\land I(x,y)) & $H(x,y)\land I(x,y)$ is always False\\
    \end{tabular}

    The existential counterparts of all the above statements also hold. In addition, the following existential statements also hold.

    \begin{tabular}{L@{ : }l}
      \exists x\exists y\; I(x,y) & $I(x,y)$ is sometimes True\\
      \exists x\exists y\; \neg I(x,y) & $I(x,y)$ is sometimes False\\
      \exists x\exists y\; (I(x,y) \iff H(x,y)) & $I(x,y)$ and $H(x,y)$ sometimes agree\\
      \exists x\exists y\; \neg (I(x,y) \iff H(x,y)) & $I(x,y)$ and $H(x,y)$ sometimes contradict\\
      \exists x\exists y\; (I(x,y) \iff G(x,y)) & $I(x,y)$ and $G(x,y)$ sometimes agree\\
      \exists x\exists y\; \neg (I(x,y) \iff G(x,y)) & $I(x,y)$ and $G(x,y)$ sometimes contradict\\
      \exists x\exists y\; (I(x,y) \land G(x,y)) & $I(x,y)\land G(x,y)$ is sometimes True\\
      \exists x\exists y\; \neg (I(x,y) \land G(x,y)) & $I(x,y)\land G(x,y)$ is sometimes False\\
      \exists x\exists y\; (I(x,y) \lor H(x,y)) & $I(x,y)\lor H(x,y)$ is sometimes True\\
      \exists x\exists y\; \neg (I(x,y) \lor H(x,y)) & $I(x,y)\lor H(x,y)$ is sometimes False\\
      \exists x\forall y\; I(x,y) & $I(x,y)$ is always True when $x=p_2$\\
      \exists x\forall y\; \neg I(x,y) & $I(x,y)$ is always False when $x=p_1$\\
    \end{tabular}
  \end{solution}
\end{questions}
\end{document}
