\documentclass[addpoints]{exam}

\usepackage{amsmath,amssymb,amsthm}
\usepackage{graphicx}
\usepackage{hyperref}
\usepackage{tabularx}

\theoremstyle{definition}
\newtheorem{definition}{Definition}[section]

\theoremstyle{claim}
\newtheorem{claim}{Claim}

\newcolumntype{L}{>$l<$}
\newcommand\Z{\ensuremath{\mathbb{Z}}}


\title{Quiz 9B: Sequences and Summation}
\author{CS/MATH 113 Discrete Mathematics L1}
\date{Habib University | Spring 2023}

\runningheader{}{}{}
\runningfooter{}{}{}

% solution
\usepackage{draftwatermark}
\SetWatermarkText{Sample Solution}
\SetWatermarkScale{3}
\printanswers

\begin{document}
\maketitle
\thispagestyle{empty}

\noindent
\begin{tabularx}{\linewidth}{Xr}
  Total Marks: \numpoints & Date: \today\\
  Duration: 10 minutes & Time: 1715--1725h
\end{tabularx}
\hrule
\bigskip

\noindent \textbf{Student ID}: \hrulefill \\[5pt]
\noindent \textbf{Student Name}: \hrulefill \\[5pt]

\section{Problems}

\begin{questions}
\question[10] Given
  \[
    f(n) = \sum_{i=1}^n k^2,
  \]
  show how you can compute $\sum_{i=10}^{100} k^2$.
 
  \begin{solution}
    \begin{align*}
      \sum_{i=10}^{100} k^2 & = \sum_{i=1}^{100} k^2 - \sum_{i=1}^{9} k^2\\
                            & = f(100) - f(9)
    \end{align*}
  \end{solution}

\end{questions}
\end{document}

%%% Local Variables:
%%% mode: latex
%%% TeX-master: t
%%% End:
