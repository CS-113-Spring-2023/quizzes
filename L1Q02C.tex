\documentclass[addpoints]{exam}

\usepackage{amsmath,amssymb,amsthm}
\usepackage{tabularx}
\usepackage{tikz}
\usetikzlibrary{graphs,graphs.standard}
\usepackage{xcolor}

\theoremstyle{definition}
\newtheorem{definition}{Definition}[section]

\theoremstyle{claim}
\newtheorem{claim}{Claim}

\title{Quiz 2C: Logical Equivalence}
\author{CS/MATH 113 Discrete Mathematics L1}
\date{Habib University | Spring 2023}

% \usepackage{draftwatermark}
% \SetWatermarkText{Sample Solution}
% \SetWatermarkScale{3}

% \printanswers

\begin{document}
\maketitle
\thispagestyle{empty}
\noindent
\begin{tabularx}{\linewidth}{Xr}
  Total Marks: \numpoints & Date: \today\\
  Duration: 10 minutes & Time: 1715--1725h
\end{tabularx}
\hrule
\bigskip

\noindent \textbf{Student ID}: \hrulefill \\[5pt]
\noindent \textbf{Student Name}: \hrulefill \\[5pt]

\section{Problems}

\begin{questions}
  \question[5] We are given the following definitions.

\begin{definition}[Conjunction, $\land$]
  \[
  \begin{array}{c|c||c}
    p & q & p \land q\\
    \hline
    T & T & T \\
    T & F & F \\
    F & T & F \\
    F & F & F \\
  \end{array}
  \]
\end{definition}

\begin{definition}[Disjunction, $\lor$]
  \[
  \begin{array}{c|c||c}
    p & q & p \lor q\\
    \hline
    T & T & T \\
    T & F & T \\
    F & T & T \\
    F & F & F \\
  \end{array}
  \]
\end{definition}

Use these to prove the following \textit{distributivity law}.
\[
  p \land (q \lor r) \equiv (p \land q) \lor (p \land r).
\]
  
\begin{solution}
  We show below that if $n$ is odd, then $n^2$ is also odd.
    \begin{proof}
      Let $n=2k+1$ where $k$ is an integer.\\
      Then $n^2=4k^2+4k+1 = 2 (2k^2+2k)+1$.\\
      $2k^2+2k$ is an integer.\\
      $\therefore n^2$ is odd.
    \end{proof}
  \end{solution}
\end{questions}
\end{document}
