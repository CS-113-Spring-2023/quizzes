\documentclass[addpoints]{exam}

\usepackage{amsmath,amssymb,amsthm}
\usepackage{graphicx}
\usepackage{hyperref}
\usepackage{tabularx}

\theoremstyle{definition}
\newtheorem{definition}{Definition}[section]

\theoremstyle{claim}
\newtheorem{claim}{Claim}

\newcolumntype{L}{>$l<$}
\newcommand\Z{\ensuremath{\mathbb{Z}}}

\title{Quiz 8B: Functions}
\author{CS/MATH 113 Discrete Mathematics L1}
\date{Habib University | Spring 2023}

\runningheader{}{}{}
\runningfooter{}{}{}

% solution
\usepackage{draftwatermark}
\SetWatermarkText{Sample Solution}
\SetWatermarkScale{3}
\printanswers

\begin{document}
\maketitle
\thispagestyle{empty}

\noindent
\begin{tabularx}{\linewidth}{Xr}
  Total Marks: \numpoints & Date: \today\\
  Duration: 10 minutes & Time: 1715--1725h
\end{tabularx}
\hrule
\bigskip

\noindent \textbf{Student ID}: \hrulefill \\[5pt]
\noindent \textbf{Student Name}: \hrulefill \\[5pt]

\section{Problems}

\begin{questions}
\question[10] Given the following definitions, prove the claim below.

  \begin{definition}[Injective function]
    A function $f$ is said to be \textit{one-to-one}, or an \textit{injection}, if and only if $f (a) = f (b)$ implies that $a = b$ for all $a$ and $b$ in the domain of $f$. A function is said to be \textit{injective} if it is one-to-one.
  \end{definition}

  \begin{definition}[Surjective function]
    A function $f$ from $A$ to $B$ is called \textit{onto}, or a \textit{surjection}, if and only if for every element $b \in B$ there is an element $a \in A$ with $f(a) = b$. A function $f$ is called \textit{surjective} if it is onto.
  \end{definition}

  \begin{definition}[Bijective function]
    The function $f$ is a \textit{one-to-one correspondence}, or a \textit{bijection}, if it is both one-to-one and onto. We also say that such a function is \textit{bijective}.
  \end{definition}

  \begin{claim}
    There are functions that are surjective but not bijective.
  \end{claim}

  \begin{solution}
    This is an existence proof. We prove the claim by providing a witness. That is, a function that is surjective and not injective.
    \begin{proof}
      Take $f:\Z\to\Z^+$ where $f(x)=|x|$.

      \underline{\textit{To show}}: $f$ is surjective.\\
      Consider $b\in\Z^+$.\\
      Then, $f(-b)=b$ and $-b\in\Z$.

      \underline{\textit{To show}}: $f$ is not injective.\\
      Consider $a=-1,b=1$.\\
      $f(a)=f(b)=1$.
    \end{proof}
  \end{solution}

\end{questions}
\end{document}

%%% Local Variables:
%%% mode: latex
%%% TeX-master: t
%%% End:
