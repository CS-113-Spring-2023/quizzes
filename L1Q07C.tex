\documentclass[addpoints]{exam}

\usepackage{amsmath,amssymb,amsthm}
\usepackage{graphicx}
\usepackage{hyperref}
\usepackage{tabularx}

\theoremstyle{definition}
\newtheorem{definition}{Definition}[section]

\theoremstyle{claim}
\newtheorem{claim}{Claim}

\newcolumntype{L}{>$l<$}
\newcommand\Z{\ensuremath{\mathbb{Z}}}

\title{Quiz 7C: Functions}
\author{CS/MATH 113 Discrete Mathematics L1}
\date{Habib University | Spring 2023}

\runningheader{}{}{}
\runningfooter{}{}{}

% solution
\usepackage{draftwatermark}
\SetWatermarkText{Sample Solution}
\SetWatermarkScale{3}
\printanswers

\begin{document}
\maketitle
\thispagestyle{empty}

\noindent
\begin{tabularx}{\linewidth}{Xr}
  Total Marks: \numpoints & Date: \today\\
  Duration: 10 minutes & Time: 1715--1725h
\end{tabularx}
\hrule
\bigskip

\noindent \textbf{Student ID}: \hrulefill \\[5pt]
\noindent \textbf{Student Name}: \hrulefill \\[5pt]

% \section{Problems}

\begin{questions}
  \question [10]

  We are given a function, $f:\Z\to \Z$, defined as $f(x) = 2x$. Argue whether the co-domain and range of $f$ are equal. Recall that these are sets.
  \begin{solution}
    The co-domain is given as \Z. Let the range be $R$. We show that $R\neq\Z$ by providing a counterexample.
    \begin{proof}
      $5\in\Z$ and $5\not\in R$.
    \end{proof}
  \end{solution}

\end{questions}
\end{document}

%%% Local Variables:
%%% mode: latex
%%% TeX-master: t
%%% End:
