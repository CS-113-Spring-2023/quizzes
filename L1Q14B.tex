\documentclass[addpoints]{exam}

\usepackage{amsmath,amssymb,amsthm}
\usepackage{tabularx}
\usepackage{tikz}
\usetikzlibrary{graphs,graphs.standard}
\usepackage{xcolor}

% solution
\usepackage{draftwatermark}
\SetWatermarkText{Sample Solution}
\SetWatermarkScale{3}
\SetWatermarkLightness{.85}
\printanswers

\theoremstyle{definition}
\newtheorem{definition}{Definition}[section]
\newtheorem{theorem}{Theorem}

\theoremstyle{claim}
\newtheorem{claim}{Claim}

\title{Quiz 14B: Graphs}
\author{CS/MATH 113 Discrete Mathematics L1}
\date{Habib University | Spring 2023}

\begin{document}
\maketitle
\thispagestyle{empty}
\noindent
\begin{tabularx}{\linewidth}{Xr}
  Total Marks: \numpoints & Date: \today\\
  Duration: 10 minutes & Time: 1530--1540h
\end{tabularx}
\hrule
\bigskip

\noindent \textbf{Student ID}: \hrulefill \\[5pt]
\noindent \textbf{Student Name}: \hrulefill \\[5pt]

\section{Problems}

\begin{questions}
\question We are given the following definitions and theorems.

\begin{definition}[Bipartite Graph]
A simple graph $G$ is called bipartite if its vertex set $V$ can be partitioned into two disjoint sets $V_1$ and $V_2$ such that every edge in the graph connects a vertex in $V_1$ and a vertex in $V_2$ (so that no edge in $G$ connects either two vertices in $V_1$ or two vertices in $V_2$). When this condition holds, we call the pair $(V_1, V_2)$ a bipartition of the vertex set $V$ of $G$.
\end{definition}

\begin{definition}[Complete Bipartite Graph]
A complete bipartite graph $K_{m,n}$ is a graph that has its vertex set partitioned into two subsets of $m$ and $n$ vertices, respectively with an edge between two vertices if and only if one vertex is in the first subset and the other vertex is in the second subset.
\end{definition}

\begin{definition}[Matching in a Simple Graph]
 A matching $M$ in a simple graph $G = (V, E)$ is a subset of the set $E$ of edges of the graph such that no two edges are incident with the same vertex.  A vertex that is the endpoint of an edge of a matching $M$ is said to be matched in $M$; otherwise it is said to be unmatched.
\end{definition}

\begin{definition}[Complete Matching in a Bipartite Graph]
We say that a matching $M$ in a bipartite graph $G = (V, E)$ with bipartition $(V_1,V_2)$ is a complete matching from $V_1$ to $V_2$ if every vertex in $V_1$ is the endpoint of an edge in the matching, or equivalently, if $|M| = |V_1|$.
\end{definition}

\begin{theorem}[Hall’s Marriage Theorem]
  The bipartite graph $G = (V, E)$ with bipartition $(V_1, V_2)$ has a complete matching from $V_1$ to $V_2$ if and only if $|N(A)| \geq |A|$ for all subsets $A$ of $V_1$.
\end{theorem}

Given $K_{3,2}$, argue whether a complete matching exists
\begin{parts}
\part[5] from $V_1$ to $V_2$,
  \begin{solution}
    We use Hall's theorem to prove that a complete matching \textit{does not exist} in this case. That is, we show a subset, $A$, of $V_1$ for which the condition in Hall's theorem does not hold.
    \begin{proof}
      \begin{itemize}
      \item We know that $|V_1| =3, |V_2| =2$, and $\forall v\in V_1 (N(v) = V_2)$.
      \item For the purpose of applying Hall's theorem, consider $A=V_1$.
      \item $N(A)=V_2$. So, $|A|=3$ and $|N(A)|=2$.
      \end{itemize}
    \end{proof}
  \end{solution}
\part[5] from $V_2$ to $V_1$.
  \begin{solution}
    We use Hall's theorem to prove that a complete matching \textit{exists} in this case.  That is, we argue that for every subset of $V_2$, the condition in Hall's theorem holds.
    \begin{proof}
      \begin{itemize}
      \item We know that $|V_2| =2, |V_1| =3$, and $\forall v\in V_2 (N(v) = V_1)$.
      \item Then, $\forall A\subseteq V_2 (N(A)=V_1)$.
      \item $A$ is largest when $A=V_2$. Then $|A|=2$ and $N(A)=3$.
      \end{itemize}
    \end{proof}
  \end{solution}
\end{parts}

\end{questions}
\end{document}

%%% Local Variables:
%%% mode: latex
%%% TeX-master: t
%%% End:
