\documentclass[addpoints]{exam}

\usepackage{amsmath,amssymb,amsthm}
\usepackage{tabularx}

\theoremstyle{definition}
\newtheorem{definition}{Definition}[section]

\theoremstyle{claim}
\newtheorem{claim}{Claim}

\title{Quiz 5C: Proofs}
\author{CS/MATH 113 Discrete Mathematics L1}
\date{Habib University | Spring 2023}

% solution
\usepackage{draftwatermark}
\SetWatermarkText{Sample Solution}
\SetWatermarkScale{3}
\printanswers

\newcolumntype{L}{>$l<$}

\begin{document}
\maketitle
\thispagestyle{empty}

\noindent
\begin{tabularx}{\linewidth}{Xr}
  Total Marks: \numpoints & Date: \today\\
  Duration: 10 minutes & Time: 1715--1725h
\end{tabularx}
\hrule
\bigskip

\noindent \textbf{Student ID}: \hrulefill \\[5pt]
\noindent \textbf{Student Name}: \hrulefill \\[5pt]

% \section{Problems}

\begin{questions}
  \question [10] Given the following proof, argue whether it is correct or not.

  \begin{claim}
  If $n^2$ is even, then so is $n$.
\end{claim}

  \begin{proof}
    Suppose that $n^2$ is even.\\
    Then $n^2 = 2k$ for some integer $k$.\\
    Let $n = 2m$ for some integer $m$.\\
    This shows that n is even.
  \end{proof}
  
  \begin{solution}
    The proof is not correct.

    The third line assumes $n=2m$. This does not follow from any of the lines above. As such, it is an assumption. If so, $n$ is no longer general.
  \end{solution}
\end{questions}
\end{document}
