\documentclass[addpoints]{exam}

\usepackage{amsmath,amssymb,amsthm}
\usepackage{tabularx}
\usepackage{tikz}
\usetikzlibrary{graphs,graphs.standard}
\usepackage{xcolor}

% solution
\usepackage{draftwatermark}
\SetWatermarkText{Sample Solution}
\SetWatermarkScale{3}
\SetWatermarkLightness{.85}
\printanswers

\theoremstyle{definition}
\newtheorem{definition}{Definition}[section]
\newtheorem{theorem}{Theorem}

\theoremstyle{claim}
\newtheorem{claim}{Claim}

\title{Quiz 13: Graphs}
\author{CS/MATH 113 Discrete Mathematics L1}
\date{Habib University | Spring 2023}

\begin{document}
\maketitle
\thispagestyle{empty}
\noindent
\begin{tabularx}{\linewidth}{Xr}
  Total Marks: \numpoints & Date: \today\\
  Duration: 10 minutes & Time: 1530--1540h
\end{tabularx}
\hrule
\bigskip

\noindent \textbf{Student ID}: \hrulefill \\[5pt]
\noindent \textbf{Student Name}: \hrulefill \\[5pt]

\section{Problems}

\begin{questions}
\question[10] Given the following definitions and theorems, prove the claim at the end.

\begin{definition}[Graph]
  A graph $G = (V, E)$ consists of $V$, a nonempty set of vertices (or nodes) and $E$, a set of edges. Each edge has either one or two vertices associated with it, called its endpoints. An edge is said to connect its endpoints.
\end{definition}

\begin{definition}[Adjacency/Incidence]
  Two vertices $u$ and $v$ in a graph $G$ are called adjacent (or neighbors) in $G$ if $u$ and $v$ are endpoints of an edge $e$ of G. Such an edge $e$ is called incident with the vertices $u$ and $v$, and $e$ is said to connect $u$ and $v$.
\end{definition}

\begin{definition}[Degree]
The degree of a vertex in a graph is the number of edges incident with it, except that a loop at a vertex contributes twice to the degree of that vertex. The degree of the vertex $v$ is denoted by $deg(v)$.
\end{definition}

\begin{theorem}[The Handshaking Theorem]
  Let $G = (V, E)$ be a graph with $m$ edges. Then
  \[
    2m = \sum_{v\in V}deg(v).
  \]
  (Note that this applies even if multiple edges and loops are present.)
\end{theorem}

\begin{claim}
  Any graph has an even number of vertices of odd degree.
\end{claim}
  
\begin{solution}
  We prove the claim directly.
  \begin{proof}
    Consider an arbitrary graph with $n$ vertices of odd degree and $m$ vertices of even degree.\\
    Let the odd degrees be: $2k_1+1, 2k_2+1, 2k_3+1, \ldots, 2k_n+1$. Their sum is: $n + 2\sum_{i=1}^nk_i$.\\
    Let the even degrees be: $2l_1, 2l_2, 2l_3, 2l_4, \ldots, 2l_m$. Their sum is: $2\sum_{i=1}^ml_i$.\\
    So, the total degree is: $n + 2\sum_{i=1}^nk_i + 2\sum_{i=1}^ml_i = n + 2K$, where $K$ is some integer.\\
    The parity of the total degree is the parity of $n + 2K$ which is the same as the parity of $n$.\\
    We know from the handshaking theorem that the total degree is even.\\
    $\therefore n$ is even.
  \end{proof}
\end{solution}
\end{questions}
\end{document}

%%% Local Variables:
%%% mode: latex
%%% TeX-master: t
%%% End:
